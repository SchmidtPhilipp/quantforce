
    \documentclass[a4paper,
        fleqn,            % Linksbündige Gleichungen
        12pt,             % Standard-Schriftgröße
        ngerman,          % Deutsche Sprache
        oneside,          % Doppelseitiges Layout
        chapterentrydots=true,  % Punkte in Inhaltsverzeichnis für Kapitel
        parskip=half      % Halber Zeilenabstand zwischen Absätzen
        ]{article}
    \usepackage{graphicx}   % Für Grafiken
    \usepackage{float}      % Für die Verwendung von [H] in \begin{figure}
    \usepackage{hyperref}   % Für Hyperlinks im PDF
    \hypersetup{
        colorlinks=true,
        linkcolor=blue,
        urlcolor=cyan,
    }
    \usepackage{booktabs}   % Für bessere Tabellen
    \usepackage{pgf}        % Für die Verwendung von .pgf-Dateien
    \usepackage{pgfplots}   % Für die Verwendung von pgfplots
    \usepackage{lmodern}    % Verbessert die Schriftart
    \usepackage{import}     % Importiere die pgf-Datei
    \def\mathdefault#1{#1}  % Verhindert Fehler bei der Verwendung von \mathdefault
    \pgfplotsset{compat=1.18}  % Setze die Kompatibilitätsversion
    \usepackage{amsmath}    % Für mathematische Formeln
    \usepackage{amssymb}    % Für mathematische Symbole
    
    \usepackage[a4paper, % Setze das Papierformat auf A4
    left=2cm, % Linker Rand
    right=2cm, % Rechter Rand
    top=1cm, % Oberer Rand
    bottom=1cm, % Unterer Rand
    includehead,  % Kopfzeilen einbeziehen
    includefoot, % Kopf- und Fußzeilen einbeziehen
    nomarginpar,% We don't want any margin paragraphs
    textwidth=10cm,% Set \textwidth to 10cm
    headheight=10mm,% Set \headheight to 10mm
    ]{geometry}
    \usepackage{fancyhdr}
    \pagestyle{fancy}
    \fancyhead{} % clear all header fields
    \fancyfoot{} % clear all footer fields
    \fancyfoot[LE,RO]{\thepage}
    \fancyhead[LE]{\leftmark} % left even page
    \fancyhead[RO]{\rightmark} % right odd page
    
            \begin{document}
    \title{Report}
    \date{\today}
    \maketitle
    \newpage
    \tableofcontents
    \newpage
    
    \begin{figure}[H]
        \centering
        \input{REPORT/portfolio_value_paper.pgf}
        \caption{}
        \label{fig:portfolio_value_paper}
    \end{figure}
    

    \begin{figure}[H]
        \centering
        \input{REPORT/portfolio_value_beamer.pgf}
        \caption{}
        \label{fig:portfolio_value_beamer}
    \end{figure}
    
\begin{table}[H]
\centering
\begin{tabular}{lcccccc}
\toprule
Metric & SPQL & PPO & TD3 & DDPG & SAC & Tangency \\
\midrule
Sharpe Ratio & 1.056 & 1.226 & 1.051 & 1.349 & 1.316 & 1.407 \\
Sortino Ratio & 0.063 & 0.069 & 0.060 & 0.075 & 0.072 & 0.080 \\
Max Drawdown & -28.13 \% & -13.28 \% & -13.00 \% & -18.76 \% & -18.90 \% & -14.17 \% \\
Annualized Volatility & 0.98 \% & 0.62 \% & 0.61 \% & 0.68 \% & 0.73 \% & 0.69 \% \\
Cumulative Return & 134.80 \% & 92.55 \% & 72.43 \% & 120.15 \% & 127.25 \% & 129.50 \% \\
Annualized Return & 19.78 \% & 14.86 \% & 12.21 \% & 18.16 \% & 18.96 \% & 19.21 \% \\
Calmar Ratio & 0.703 & 1.119 & 0.939 & 0.968 & 1.003 & 1.356 \\
\bottomrule
\end{tabular}
\caption{Metrics Table}
\label{tab:metrics_table}
\end{table}
\begin{table}[H]
\centering
\begin{tabular}{|l|c|c|c|c|c|c|c|}
\toprule
\rotatebox{90}{Name} & \rotatebox{90}{Sharpe Ratio} & \rotatebox{90}{Sortino Ratio} & \rotatebox{90}{Max Drawdown} & \rotatebox{90}{Annualized Volatility} & \rotatebox{90}{Cumulative Return} & \rotatebox{90}{Annualized Return} & \rotatebox{90}{Calmar Ratio} \\
\midrule
SPQL & 1.056 & 0.063 & -28.13 \% & 0.98 \% & 134.80 \% & 19.78 \% & 0.703 \\
PPO & 1.226 & 0.069 & -13.28 \% & 0.62 \% & 92.55 \% & 14.86 \% & 1.119 \\
TD3 & 1.051 & 0.060 & -13.00 \% & 0.61 \% & 72.43 \% & 12.21 \% & 0.939 \\
DDPG & 1.349 & 0.075 & -18.76 \% & 0.68 \% & 120.15 \% & 18.16 \% & 0.968 \\
SAC & 1.316 & 0.072 & -18.90 \% & 0.73 \% & 127.25 \% & 18.96 \% & 1.003 \\
Tangency & 1.407 & 0.080 & -14.17 \% & 0.69 \% & 129.50 \% & 19.21 \% & 1.356 \\
\bottomrule
\end{tabular}
\caption{Metrics Table (Transposed)}
\label{tab:metrics_table_transposed}
\end{table}
    \end{document}
    