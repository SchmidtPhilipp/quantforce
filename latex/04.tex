\section{Rechtliche Risiken und Haftung für Unternehmen}
%
\subsection{Klimahaftung und Greenwashing-Vorwürfe}
%
Unternehmen sehen sich vermehrt mit rechtlichen Risiken im Zusammenhang mit ihren Umweltauswirkungen konfrontiert. Die {umwelthaftungsrichtlinie} verpflichtet Betreiber bestimmter beruflicher Tätigkeiten zur Vermeidung und Sanierung von Umweltschäden\footcite[Art. 3-6]{eu_umwelthaftung}.
% „Gemäß der Richtlinie 2004/35/EG über Umwelthaftung sind Betreiber bestimmter beruflicher Tätigkeiten für Umweltschäden verantwortlich und müssen Präventions- und Sanierungsmaßnahmen ergreifen.“ (Art. 3, 5 und 6 der RL 2004/35/EG)  
%
Die Haftung von Unternehmen im Umweltbereich ist dabei nicht auf unmittelbare Verstöße gegen nationale oder europäische Vorschriften beschränkt, sondern umfasst auch indirekte Auswirkungen ihrer Geschäftstätigkeit auf Klima und Umwelt\footcite[]{eu_nature_restoration}.  
% „Die EU strebt mit der Verordnung zur Wiederherstellung der Natur eine stärkere Kontrolle über umweltbezogene Verpflichtungen von Unternehmen an.“ (environment.ec.europa.eu)  
%
Zudem nimmt die Anzahl der Klimaklagen gegen Unternehmen zu. In der Rechtssache C-565/19 P (Armando Carvalho u. a./Europäisches Parlament und Rat der Europäischen Union) bestätigte der Europäische Gerichtshof, dass Einzelpersonen nur dann eine direkte Klimaklage gegen EU-Institutionen führen können, wenn eine individuelle Betroffenheit nachgewiesen wird\footcite[]{eu_klimaklage}.  
% „Der EuGH bestätigte die Unzulässigkeit der Klimaklage Carvalho u. a. gegen die EU, da keine individuelle Betroffenheit festgestellt wurde.“ (EuGH Rs. C-565/19)  
%
Während diese Entscheidung die gerichtliche Durchsetzbarkeit von Klimaschutzverpflichtungen einschränkt, verdeutlicht sie zugleich, dass indirekte Verantwortlichkeit von Unternehmen für Klimaschäden zunehmend ein Thema gerichtlicher Auseinandersetzungen wird. Auch nationale Gerichte haben in jüngerer Zeit Unternehmen für ihre Umweltauswirkungen zur Verantwortung gezogen, insbesondere im Zusammenhang mit Menschenrechten und ökologischer Sorgfaltspflicht\footcite[]{eu_csddd}.  
% „Unternehmen sind verpflichtet, ihre Geschäftstätigkeit und Lieferketten auf nachteilige Auswirkungen für Umwelt und Menschenrechte zu überprüfen.“ (CSDDD, Art. 4)  
%
Unternehmen müssen daher nicht nur regulatorische Vorgaben einhalten, sondern auch reputationsbezogene Risiken wie Greenwashing-Vorwürfe vermeiden. Falsche oder irreführende Angaben zu Nachhaltigkeitsstrategien können zu wettbewerbsrechtlichen Konsequenzen führen und Haftungsansprüche nach sich ziehen\footcite[]{eu_taxonomie}.
%