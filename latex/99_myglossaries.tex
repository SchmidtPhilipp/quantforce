
\newglossary[slg]{symbols}{sym}{sbl}{Mathematical Symbols} % define a list of symbols
\newglossary[mth]{math}{mat}{mtt}{Special Mathematical Operations} % define a list of symbols

%\newglossary[gls]{glossar}{glo}{gls}{Glossar} % define a list of abbreviations

% prevent hyphenating abbreviations
\renewcommand\newacronym[3]{\hyphenation{#2}\oldnewacronym{#1}{#2}{#3}}

% the default width of the description column looks odd when using KOMA
% http://www.mrunix.de/forums/showthread.php?65790-Eine-Frage-guten-Stils-Glossaries-KOMA-(scrbook)-und-style-list
\setlength{\glsdescwidth}{0.7\textwidth}

% define a glossarystyle with linebrakes
% http://tex.stackexchange.com/questions/203718/manual-line-break-in-glossaries-name
\newlength\glsnamewidth
\setlength{\glsnamewidth}{0.2\textwidth}
\newglossarystyle{superglossarystyle}
{%
	\setglossarystyle{super}%
	\renewenvironment{theglossary}%
	{%
		\tablehead{}%
		\tabletail{}%
		%\begin{supertabular}{p{\glsnamewidth}p{\glsdescwidth}}%
		\begin{longtable}{p{\glsnamewidth}p{\glsdescwidth}}%
			
			%\begin{supertabular}{p{\glsnamewidth}p{\textwidth-2\tabcolsep-\glsnamewidth}}%
		}%
		{%
			%\end{supertabular}%
		\end{longtable}%
	}%
	\renewcommand{\glossentry}[2]{%
		\raggedleft
		\glsentryitem{##1}\glstarget{##1}{\glossentryname{##1}} &
		\glossentrydesc{##1}\glspostdescription\space ##2\tabularnewline
	}%
}
\setglossarystyle{superglossarystyle}
%\makeglossaries % produces the *.ist file for makeindex.exe or makeglossaries


% Fachbegriffe

\newglossaryentry{reinforcement_learning}{
	name={Reinforcement Learning},
	description={Ein Bereich des maschinellen Lernens, in dem ein Agent durch Interaktion mit einer Umgebung eine optimale Strategie erlernt, um eine langfristige Belohnung zu maximieren}
}

\newglossaryentry{multi_agent_reinforcement_learning}{
	name={Multi-Agent Reinforcement Learning},
	description={Eine Erweiterung des Reinforcement Learning, bei der mehrere Agenten simultan in einer Umgebung agieren und interagieren}
}

\newglossaryentry{black_scholes_modell}{
	name={Black-Scholes-Modell},
	description={Ein Modell zur Bewertung von Derivaten, das auf der Annahme einer geometrischen Brown’schen Bewegung basiert}
}

\newglossaryentry{markov_decision_process}{
	name={Markov-Entscheidungsprozess},
	description={Mathematische Modellierung eines Entscheidungsproblems mit Zuständen, Aktionen, Belohnungen und Übergangswahrscheinlichkeiten}
}

\newglossaryentry{policy}{
	name={Policy},
	description={Strategie eines RL-Agenten, die beschreibt, welche Aktion in einem bestimmten Zustand ausgeführt wird}
}

\newglossaryentry{ml}{
	name={Machine Learning},
	description={Ein Teilgebiet der künstlichen Intelligenz, das sich mit der Entwicklung von Algorithmen und Modellen befasst, die es Computern ermöglichen, aus Daten zu lernen}
}




