\section{Einleitung}
\label{sec:einleitung}

Die zunehmende Automatisierung und Digitalisierung der Finanzmärkte hat den Einsatz von \gls{reinforcement_learning} (RL) als vielversprechende Methode zur Optimierung algorithmischer Handelsstrategien verstärkt. Insbesondere \gls{multi_agent_reinforcement_learning} (MARL) hat sich als ein leistungsfähiges Konzept etabliert, um Handelsstrategien durch kooperative und kompetitive Interaktionen autonomer Agenten zu verbessern\cite{Lowe2017}. Durch den Einsatz moderner RL-Methoden lassen sich komplexe Marktstrukturen modellieren und adaptive Strategien entwickeln, die über traditionelle regelbasierte Ansätze hinausgehen\cite{Zhang2023}.

\subsection{Motivation und Problemstellung}

Klassische Handelsstrategien basieren oft auf starren Regeln oder heuristischen Verfahren, die auf historische Daten kalibriert wurden. Diese Methoden stoßen an ihre Grenzen, wenn Marktbedingungen sich abrupt ändern oder unvorhersehbare Ereignisse auftreten. \gls{reinforcement_learning} ermöglicht es, Strategien kontinuierlich anzupassen, indem Agenten durch Interaktionen mit dem Marktumfeld lernen, optimale Entscheidungen zu treffen\cite{Thakkar2021}. Dabei zeigt insbesondere \gls{multi_agent_reinforcement_learning} vielversprechende Ansätze für den algorithmischen Handel und das Portfoliomanagement, da mehrere Agenten simultan agieren und sich gegenseitig beeinflussen\cite{Lowe2017}.

\subsection{Forschungsfrage und Zielsetzung}

Diese Arbeit untersucht die Frage:  
*"Kann Multi-Agent Reinforcement Learning klassische Portfoliomanagement-Ansätze übertreffen?"*  

Zur Beantwortung dieser Frage werden bestehende Methoden aus der Finanzmathematik mit modernen MARL-Ansätzen verglichen. Es wird analysiert, inwiefern RL-Strategien klassische Optimierungsmodelle wie das \gls{markowitz_portfolio_theorie} oder regelbasierte Handelsstrategien in Bezug auf Rendite und Risikomanagement übertreffen können.

\subsection{Methodik und Aufbau der Arbeit}

Zur Untersuchung dieser Fragestellung wird eine umfassende Literaturanalyse durchgeführt, die aktuelle Forschungsergebnisse zu MARL und Finanzmärkten systematisch bewertet. Dabei werden sowohl theoretische Grundlagen als auch empirische Studien betrachtet, um eine fundierte Einschätzung der Leistungsfähigkeit von MARL zu erhalten.

Die Arbeit gliedert sich wie folgt:
Kapitel~\ref{sec:grundlagen} behandelt die theoretischen Grundlagen von RL und MARL, einschließlich mathematischer Modelle und Algorithmen. In Kapitel~\ref{sec:anwendungen} werden Anwendungsfälle von MARL in Finanzmärkten diskutiert. Kapitel~\ref{sec:vergleich} vergleicht klassische Finanzmodelle mit MARL-basierten Handelsstrategien. Anschließend werden in Kapitel~\ref{sec:kritische_diskussion} die Herausforderungen und Limitationen von MARL untersucht. Kapitel~\ref{sec:fazit} fasst die Ergebnisse zusammen und gibt einen Ausblick auf zukünftige Entwicklungen.
