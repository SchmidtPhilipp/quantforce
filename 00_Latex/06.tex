\section{Ausblick und Fazit}

\subsection{Zukunftstrends im Klimarecht}

Die rechtlichen Anforderungen im Bereich des Klimaschutzes werden in den kommenden Jahren weiter verschärft. Die Europäische Union verfolgt mit ihrer Klimastrategie eine zunehmende Verpflichtung zur Emissionsreduktion, die sich insbesondere durch die Ausweitung bestehender Emissionshandelssysteme (EU-ETS) und durch neue Berichtspflichten für Unternehmen manifestiert\footcite[]{eu_klimapolitik}.  
% „Die EU verfolgt mit dem europäischen Klimagesetz verbindliche Klimaneutralitätsziele, die schrittweise durch regulatorische Vorgaben ergänzt werden.“ (consilium.europa.eu)  

Ein zentrales Element dieser Entwicklung ist die Corporate Sustainability Due Diligence Directive (CSDDD), die Unternehmen verpflichtet, Umwelt- und Menschenrechtsrisiken entlang der gesamten Lieferkette systematisch zu identifizieren und zu minimieren\footcite[]{eu_csddd}.  
% „Unternehmen müssen Nachhaltigkeitsrisiken proaktiv bewerten und Maßnahmen zur Schadensbegrenzung implementieren.“ (CSDDD, Art. 4)  
%
Parallel dazu gewinnt die Umwelthaftung an Bedeutung. Mit der zunehmenden Anzahl von Klimaklagen auf nationaler und europäischer Ebene wächst der rechtliche Druck auf Unternehmen, nachweisbare Nachhaltigkeitsstrategien zu entwickeln\footcite[]{eu_umwelthaftung}.  
% „Die Umwelthaftungsrichtlinie 2004/35/EG regelt die Verantwortung von Unternehmen für Umweltschäden und die Pflicht zur Sanierung solcher Schäden.“ (RL 2004/35/EG, Art. 3-6)  
%
\subsection{Handlungsempfehlungen für Unternehmen}

Angesichts der steigenden regulatorischen Anforderungen sollten Unternehmen frühzeitig rechtssichere Nachhaltigkeitsstrategien implementieren. Dazu gehören insbesondere:
\begin{itemize}
	\item Verbindliche CO\textsubscript{2}-Reduktionspläne gemäß den Anforderungen des EU-Klimarechts\footcite[]{eu_klimapolitik}.
	\item Integration von ESG-Risiken in die Unternehmensstrategie, um Greenwashing-Vorwürfe und Haftungsrisiken zu vermeiden\footcite[]{bafin_nachhaltigkeit}.
	\item Transparente und belastbare Nachhaltigkeitsberichterstattung gemäß der \gls{CSRD}, um rechtliche Unsicherheiten zu minimieren\footcite[]{eu_csrd}.
\end{itemize}

Unternehmen, die diese Maßnahmen frühzeitig umsetzen, können nicht nur regulatorische Risiken minimieren, sondern auch Wettbewerbsvorteile erzielen.

\subsection{Zusammenfassung der Ergebnisse}

Die rechtlichen Rahmenbedingungen für den Klimaschutz haben sich in den letzten Jahren erheblich verändert. Während Unternehmen früher weitgehend auf freiwillige Maßnahmen setzen konnten, sind sie heute durch eine Vielzahl von EU-Verordnungen und nationalen Gesetzen zur Einhaltung verbindlicher Umweltstandards verpflichtet\footcite[]{eu_klimapolitik}.  
% „Die EU hat in den vergangenen Jahren verbindliche Klimaschutzmaßnahmen etabliert, die von Unternehmen einzuhalten sind.“ (consilium.europa.eu)  
%
Die zunehmende Verrechtlichung des Klimaschutzes zeigt sich auch in der Erweiterung der Berichtspflichten sowie in der Haftung für Umweltschäden\footcite[]{eu_umwelthaftung}.  
% „Mit der Umwelthaftungsrichtlinie 2004/35/EG sind Unternehmen verpflichtet, Umweltschäden zu vermeiden und im Schadensfall Sanierungsmaßnahmen zu ergreifen.“ (RL 2004/35/EG, Art. 5)  
%
Unternehmen, die Nachhaltigkeitsaspekte als integralen Bestandteil ihrer Geschäftsstrategie begreifen, werden nicht nur rechtliche Sicherheit gewinnen, sondern auch langfristig wirtschaftliche Vorteile erzielen\footcite[]{eu_csrd}.  
% „Die CSRD erweitert die Berichtspflichten, um Transparenz über ESG-Faktoren zu schaffen und nachhaltige Investitionen zu fördern.“ (RL (EU) 2022/2464)  

