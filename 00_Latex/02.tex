\section{Grundlagen der Quantitativen Finanzwissenschaft}
\label{sec:grundlagen}

Die quantitative Finanzwissenschaft befasst sich mit der Anwendung mathematischer und statistischer Methoden zur Modellierung und Analyse von Finanzmärkten. Ein wesentliches Ziel besteht darin, Marktbewegungen vorherzusagen, Risikomanagementstrategien zu optimieren und Handelsstrategien algorithmisch zu verbessern\footcite{Shreve2004}.

\subsection{Mathematische Modellierung in der Finanzwelt}
Mathematische Modelle spielen eine zentrale Rolle in der Finanzwissenschaft. Sie ermöglichen die formale Beschreibung von Preisbewegungen, Risikomaßen und Optimierungsproblemen. Ein klassisches Modell ist das \gls{black_scholes_modell}, das zur Bewertung von Optionen verwendet wird\footcite{BlackScholes1973}.

\subsection{Stochastische Prozesse und Finanzmathematik}
Die Modellierung von Finanzmärkten erfordert den Einsatz stochastischer Prozesse, da Kursbewegungen von Unsicherheit und zufälligen Schwankungen geprägt sind. Wichtige Prozesse in diesem Zusammenhang sind:

\subsubsection{Wiener-Prozesse und Brown’sche Bewegung}
Die Brown’sche Bewegung, eingeführt durch \cite{Bachelier1900}, bildet die Grundlage vieler stochastischer Finanzmodelle. Der Wiener-Prozess \( W_t \) ist ein spezieller Fall und erfüllt die Eigenschaften:

\begin{equation}
	W_t \sim \mathcal{N}(0, t)
\end{equation}

\subsubsection{Martingale und arbitragefreie Märkte}
Ein Martingal ist ein stochastischer Prozess \( X_t \), für den gilt:

\begin{equation}
	\mathbb{E}[X_{t+1} \mid X_t, X_{t-1}, \dots] = X_t
\end{equation}

Dies bedeutet, dass die beste Schätzung für den zukünftigen Wert eines Martingals sein aktueller Wert ist\footcite{FollmerSchied2016}. Martingale sind eng mit arbitragefreien Märkten verbunden, da unter der Martingal-Maß keine risikolosen Arbitragemöglichkeiten existieren.

\subsection{Klassische Finanzmodelle}
Einige der bekanntesten Modelle in der Finanzmathematik sind:

\begin{itemize}
	\item \textbf{Black-Scholes-Merton-Modell}: Zur Bewertung von Optionen unter der Annahme von stochastischer Volatilität\footcite{BlackScholes1973}.
	\item \textbf{Heston-Modell}: Eine Erweiterung des Black-Scholes-Modells, das eine zufällig schwankende Volatilität berücksichtigt\footcite{Heston1993}.
	\item \textbf{Markowitz-Portfolio-Theorie}: Ein fundamentales Modell zur Portfolio-Optimierung basierend auf Erwartungswerten und Kovarianzen von Wertpapieren\footcite{Markowitz1952}.
\end{itemize}

\subsection{Herausforderungen klassischer Finanzmodelle}
Klassische Finanzmodelle stoßen in realen Märkten an ihre Grenzen. Sie beruhen oft auf Annahmen wie der Normalverteilung von Renditen oder der Existenz eines risikofreien Zinssatzes. In der Praxis zeigen sich jedoch:

\begin{itemize}
	\item \textbf{Fette Ränder in Renditeverteilungen}: Extremwerte treten häufiger auf als in der Normalverteilung.
	\item \textbf{Mangelnde Berücksichtigung von Marktstrukturen}: Mikrostrukturen des Marktes wie Liquiditätsengpässe oder algorithmischer Handel werden oft nicht berücksichtigt.
	\item \textbf{Nichtstationarität der Finanzzeitreihen}: Finanzmärkte sind dynamisch, sodass Modelle kontinuierlich angepasst werden müssen.
\end{itemize}
