%%%%%%%%%%%%%%%%%%%%%%%%%%%%%%%%%%%%%%%%%%%%%%%%%%%%%%%%%%%%%%%%%%%%%%%%%%%%%%%%
%% 
%% Abstract of your thesis in default document language
%% 
\cleardoubleoddpage%  Make sure to start abstract on a new odd page
\includeChapterImage
\begin{abstract*}
	Autonomous vehicles have made significant progress towards achieving fully autonomous driving thanks to advances in \gls{radar}, vision, \gls{lidar}, and also thanks to advances in \gls{ml}. 
	\Gls{radar} is one of the most essential elements for overcoming this challenge as it allows monitoring the environment even in extreme conditions such as fog, rain, and snowfall. 
	However, to ensure the reliability and accuracy of \gls{radar} systems, the devices must be tested before delivery, but
	until now, the functionality of \gls{radar} circuits at frequencies between \SI{76}{\giga\hertz} and \SI{77}{\giga\hertz} could not be verified by external measures.
	The primary root cause was that the test connections between the test equipment and the device were insufficiently reproducible.
	New technologies such as \gls{AFiP} have now improved the reproducibility of measurements to such an extent that it seems possible to check the functionality of \gls{radar} circuits using external signal processing.
	
	In this work, a \gls{radar} range test system is developed that can be mounted onto an \gls{ATE} and allows testing up to sixteen devices simultaneously due to its small form factor. 
	The system was designed using \gls{rwg} as the main transmission line. 
	One of the system's key pieces is a multi-hole directional coupler, which attenuates the transmit signal to avoid overloading the receiver. 
	A second essential component is the delay line that delays the signal by an equivalent of \SI{4}{\meter} in air. 
	\Gls{cover-to-choke} flanges were employed to improve the electrical connection.
	In addition, many auxiliary components were designed to build the entire system. 
	The design was verified, and defective devices were successfully detected among functional ones. 
\end{abstract*} 
%% 
%%%%%%%%%%%%%%%%%%%%%%%%%%%%%%%%%%%%%%%%%%%%%%%%%%%%%%%%%%%%%%%%%%%%%%%%%%%%%%%%


%%%%%%%%%%%%%%%%%%%%%%%%%%%%%%%%%%%%%%%%%%%%%%%%%%%%%%%%%%%%%%%%%%%%%%%%%%%%%%%%
%% 
%% Abstract of your thesis in an additional language
%% 
\cleardoubleoddpage%  Make sure to start abstract on a new odd page
\includeChapterImage
\begin{otherlanguage}{ngerman}
\begin{abstract*}
	Moderne autonome Fahrzeuge stehen dank der Fortschritte in \gls{germanradar}, Vision, \gls*{germanlidar} und auch dank der Fortschritte in \gls{germanml} kurz vor dem autonomen Fahren. 
	\Gls{germanradar} ist eines der wichtigsten Elemente um diese Herausforderung zu meistern, da es ermöglicht, die Umgebung auch unter extremen Bedingungen wie Nebel, Regen und Schneefall zu überwachen. 
	Um die Zuverlässigkeit und Genauigkeit von Radarsystemen zu gewährleisten, müssen die Geräte jedoch vor der Auslieferung getestet werden.
	Bislang konnte die Funktionsfähigkeit von Radarschaltungen bei Frequenzen zwischen \SI{76}{\giga\hertz} und \SI{77}{\giga\hertz} nicht durch externe Maßnahmen überprüft werden.
	Die Hauptursache war, dass die Testverbindungen zwischen der Testausrüstung und dem Gerät nicht ausreichend reproduzierbar waren.
	Neue Technologien wie \gls{germanAFiP} haben die Reproduzierbarkeit von Messungen inzwischen so weit verbessert, dass es möglich scheint, 
	die Funktionsfähigkeit von Radarschaltungen durch externe Signalverarbeitung zu überprüfen.

	In dieser Arbeit wurde ein Testsystem für \gls{germanradar} \glspl{germanic} entwickelt, welches auf einem \gls{germanATE} montiert werden kann.  
	Aufgrund des geringen Formfaktors können auf einem \gls{germanATE} bis zu sechzehn Testsysteme montiert werden und damit \gls{germanradar} \glspl{germanic} gleichzeitig getestet werden. 
	Das System wurde unter Verwendung von \glspl{germanrwg} als Hauptübertragungsleitung konzipiert. 
	Ein Herzstück des Systems ist ein Richtkoppler, welcher genutzt wird, um das Sendesignal zu dämpfen und eine Überlastung des Empfängers zu vermeiden. 
	Eine zweite sehr wichtige Komponente ist die Verzögerungsleitung, welche das Signal um ein Äquivalent von \SI{4}{\meter} in Luft verzögert. 
	Cover-to-Choke-Flansche wurden eingesetzt, um die elektrische Verbindung zu verbessern, und viele Nebenkomponenten wurden entworfen, um das gesamte System aufzubauen. 
	Im Anschluss wurde das Design verifiziert und es wurde gezeigt, dass defekte Geräte unter Funktionalen erfolgreich erkannt werden können.
\end{abstract*}
\end{otherlanguage}
%% 
%%%%%%%%%%%%%%%%%%%%%%%%%%%%%%%%%%%%%%%%%%%%%%%%%%%%%%%%%%%%%%%%%%%%%%%%%%%%%%%